\documentclass[12pt,a4paper,titlepage,headinclude,bibtotoc]{scrartcl}

%---- Allgemeine Layout Einstellungen ------------------------------------------

% Für Kopf und Fußzeilen, siehe auch KOMA-Skript Doku
\usepackage[komastyle]{scrpage2}
\pagestyle{empty}
\setheadsepline{0.5pt}[\color{black}]
\automark[section]{chapter}


%Einstellungen für Figuren- und Tabellenbeschriftungen
\setkomafont{captionlabel}{\sffamily\bfseries}
\setcapindent{0em}


%---- Weitere Pakete -----------------------------------------------------------
% Die Pakete sind alle in der TeX Live Distribution enthalten. Wichtige Adressen
% www.ctan.org, www.dante.de

% Sprachunterstützung
\usepackage[ngerman]{babel}

% Benutzung von Umlauten direkt im Text
% entweder "latin1" oder "utf8"
\usepackage[utf8]{inputenc}

% Pakete mit Mathesymbolen und zur Beseitigung von Schwächen der Mathe-Umgebung
\usepackage{latexsym,exscale,stmaryrd,amssymb,amsmath}


\usepackage[nointegrals]{wasysym}
\usepackage{eurosym}

% Anderes Literaturverzeichnisformat
%\usepackage[square,sort&compress]{natbib}
\usepackage{hyperref}
% Für Farbe
\usepackage{color}
\usepackage{graphicx}
\usepackage{wrapfig}
\usepackage{subfigure}

% Caption neben Abbildung
\usepackage{sidecap}

% Befehl für "Entspricht"-Zeichen
\newcommand{\corresponds}{\ensuremath{\mathrel{\widehat{=}}}}
% Befehl für Errorfunction
\newcommand{\erf}[1]{\text{ erf}\ensuremath{\left( #1 \right)}}

%Fußnoten zwingend auf diese Seite setzen
\interfootnotelinepenalty=1000

%Für chemische Formeln (von www.dante.de)
%% Anpassung an LaTeX(2e) von Bernd Raichle
\makeatletter
\DeclareRobustCommand{\chemical}[1]{%
  {\(\m@th
   \edef\resetfontdimens{\noexpand\)%
       \fontdimen16\textfont2=\the\fontdimen16\textfont2
       \fontdimen17\textfont2=\the\fontdimen17\textfont2\relax}%
   \fontdimen16\textfont2=2.7pt \fontdimen17\textfont2=2.7pt
   \mathrm{#1}%
   \resetfontdimens}}
\makeatother

%Honecker-Kasten mit $$\shadowbox{$xxxx$}$$
\usepackage{fancybox}

%SI-Package
\usepackage{siunitx}

%keine Einrückung, wenn Latex doppelte Leerzeile
\parindent0pt

%Bibliography \bibliography{literatur} und \cite{gerthsen}
%\usepackage{cite}
\usepackage{babelbib}
\selectbiblanguage{ngerman}

\begin{document}

\begin{titlepage}
\centering
\textsc{\Large Praktikum zur Einführung in die physikalische Chemie,\\[1.5ex] Universität Göttingen}

\vspace*{2cm}

\rule{\textwidth}{1pt}\\[0.5cm]
{\huge \bfseries
  V5: Leitfähigkeit\\[1.5ex]
  wässriger Elektrolyte}\\[0.5cm]
\rule{\textwidth}{1pt}

\vspace*{1cm}


\begin{Large}
\begin{tabular}{ll}
Durchführende: &  Alea Tokita, Julia Stachowiak\\
Assistentin: & Annemarie Kehl\\
 Versuchsdatum: & 01.02.2016\\
 Datum der ersten Abgabe: & 08.02.2016\\

\end{tabular}
\end{Large}

\vspace*{1.5cm}

\begin{Large}
\fbox{
  \begin{minipage}[t][5cm][t]{7cm} 
  Messwerte:\\
  
Literaturwert: \\

  
  
  \end{minipage}
}
\end{Large}

\end{titlepage}

\tableofcontents

\newpage






\section{Auswertung}

Aus den Messungen werden die Mittelwerte des Leitwertes $L$ bestimmt und die Eigenleitfähigkeit des Wassers davon abgezogen. Mit der Zellkonstante $Z$ der Leitfähigkeits-Messzelle wird die spezifische Leitfähigkeit $\kappa$ für jede Lösung errechnet:\\

\begin{equation}
\kappa = \frac{Z}{R} = Z \cdot L
\end{equation}

Daraus ergibt sich die molare Leitfähigkeit $\Lambda$ der Lösungen:\\

\begin{equation}
\Lambda = \frac{\kappa}{c^*}
\end{equation}

Für die Auftragungen wird die molare Leitfähigkeit bei der gemessenen Temperatur auf die Leitfähigkeit bei $25^\circ\text{C}$ umgerechnet, der Koeffizient $m$ ist für die beiden Lösungen unterschiedlich:\\

\begin{equation}
\Lambda (25^\circ\text{C}) = \Lambda(\Omega) \cdot [1+ m \cdot (25- (\Omega/^\circ\text{C}))]
\end{equation}\\

{\centering
$m_{\mathrm{KCl}} = 2,31 \cdot 10^{-2}$ für $0,1\, \mathrm{M} > c_s > 0,001\, \mathrm{M}$\\
$m_{\mathrm{HAc}} = 1,44 \cdot 10^{-2}$ für $0,1\, \mathrm{M} > c_s > 0,001\, \mathrm{M}$\\}



\subsection{Bestimmung von $\Lambda^0$ und $K_{\mathrm{S}}$ für Essigsäure}

Für den schwachen Elektrolyten kann das Ostwaldsche Verdünnungsgesetz umgeformt werden:\\

\begin{equation}
\frac{1}{\Lambda} = \frac{1}{\Lambda^0} + \frac{c^* \cdot \Lambda}{K_{\mathrm{S}} \cdot (\Lambda^0)^2 \cdot c^0}
\end{equation}

Aufgetragen wird $\frac{1}{\Lambda}$ gegen $\frac{c^* \cdot \Lambda}{c^0}$.
Der reziproke Wert für die Grenzleitfähigkeit $\Lambda^0$ ergibt somit durch Extrapolation des Graphen als Schnittpunkt mit der Abszisse.\\
Als Steigung $m$ bleibt  $m = \frac{1}{K_{\mathrm{S}} \cdot (\Lambda^0)^2 }$.\\
Die Säurekonstante $K_{\mathrm{S}}$ errechnet sich damit folgendermaßen:\\

\begin{equation}
K_{\mathrm{S}} = \frac{1}{m \cdot (\Lambda^0)^2}
\end{equation}

Folgende Werte ergeben sich für die Auftragung:\\

\begin{table} [h]
\centering 
\begin{tabular}{|p{4cm}||p{4cm}|p{4cm}|p{4cm}|}
\hline
& $\frac{1}{\Lambda(25^\circ\text{C})}$ in $\frac{\mathrm{mol}}{\mathrm{S} \cdot \mathrm{cm}}$ & $\frac{c^*}{c^0} \cdot \Lambda(25^\circ\text{C})$ in $\frac{\mathrm{S} \cdot \mathrm{cm}}{\mathrm{mol}}$ & $\frac{1}{\Lambda}$ in $\frac{\mathrm{mol}}{\mathrm{S} \cdot \mathrm{cm}}$\\
\hline
0,1 M & & & \\
\hline
0,01 M & & & \\
\hline
0,001 M & & & \\
\hline
\end{tabular}
\end{table}

Daraus ergibt sich:\\

$K_{\mathrm{S}} = 2,5 \cdot 10^{-5}$\\
$\Lambda^0 = 0,3\, \frac{\mathrm{mol}}{\mathrm{S} \cdot \mathrm{cm}}$\\





\subsection{Bestimmung von $\Lambda^0$ für Kaliumchlorid}

Für den starken Elektrolyten Kaliumchlorid wird das Kohlrausche Quadratwurzelgesetz $\Lambda$ gegen $\sqrt{c}$ aufgetragen:\\

\begin{equation}
\Lambda = \Lambda^0 - k \cdot \sqrt{c}
\end{equation}

$\Lambda^0$ ergibt sich ebenfalls aus Extrapolation als Schnittpunkt mit der Abszisse.

Für die Auftragung ergeben sich als Werte:\\


\begin{table} [h]
\centering 
\begin{tabular}{|p{4cm}||p{4cm}|p{4cm}|}
\hline
& $\Lambda(25^\circ\text{C})$ in $\frac{\mathrm{S} \cdot \mathrm{cm}}{\mathrm{mol}}$ & $\sqrt{c}$ in $\mathrm{mol^{\frac{1}{2}}} \cdot \mathrm{l^{- \frac{1}{2}}}$\\
\hline
0,1 M & &  \\
\hline
0,01 M & &  \\
\hline
0,001 M & &  \\
\hline
\end{tabular}
\end{table}

Für $\Lambda^0$ ergibt sich somit:\\
$\Lambda^0 = 15$\\

\section{Fehlerrechnung}

\subsection{absolute Fehler}
Die absoluten Fehler bzw. Messungenauigkeiten der Geräte betragen:\\

\begin{table} [h]
\centering 
\begin{tabular}{p{4cm}p{4cm}}
$\Delta$ Temperatur & $= 0,1^\circ\text{C}$ \\
$\Delta$ Kolben &$= 1$ mL   \\
$\Delta$ Pipette &  $=0,1$ mL \\
\end{tabular}
\end{table}





\subsection{Fehlerrechnung}
Zuerst wird die absolute Standartabweichung der Leitwerte nach folgender Formel bestimmt:\\

\begin{equation}
s_{\mathrm{N}} = \sqrt{\frac{1}{N-1} \sum_{i=1}^{N}(x_i -\bar{x})}
\end{equation} 

Da es sich um sehr wenige Werte handelt (jeweils 5), muss die Standartabweichung noch mit dem Student'schen t-Faktor multipliziert werden, um den Fehler für $\bar{L}$ zu erhalten:\\

\begin{equation}
\Delta \bar{L} = t_N \cdot \bar{s}_N
\end{equation}

Für $95,5 \%$ Konfidenz und 5 Messwerte beträgt dieser 2,8\protect\footnotemark

\footnotetext{Götz, Eckold: \emph{Grundbegriffe der Fehleranalyse bei praktischen Messungen}, Institut für physikalische Chemie, Uni Göttingen, \textbf{2015}.}


Somit ergeben sich folgende Fehler für $\bar{L}$:

\begin{table} [h]
\centering 
\begin{tabular}{|p{4cm}||p{2cm}|p{2cm}|p{2cm}|}
\hline
& 0,1 M & 0,01 M & 0,001 M \\
\hline
Essigsäure & & & \\
$\bar{L}$ &7,46 & 2,346 & 6,99\\
$s_N$ & 0,055 & 0,019 & 0,11 \\
$\Delta L$ & 0,15& 0,054& 0,31\\
\hline
Kaliumchlorid & & &\\
$\bar{L}$ & 1,756 & 1,98 & 2,04\\
$s_N$& 0,017 & 0,017 & 0,025\\
$\Delta L$ & 0,047& 0,047& 0,070\\
\hline
\end{tabular}
\end{table}

\subsection{Fehlerfortpflanzung für $\Lambda$}
$\Lambda$ wird in den Rechnungen weiterverwandt und aufgetragen, sodass eine Fehlerfortpflanzung nach Gauß durchgeführt werden muss. Die Formel dafür lautet:\\

\begin{equation}
\Delta f = \sqrt{\sum_i \left(\frac{\delta f}{\delta x_i}\right)^2_j \cdot \Delta x_i^2}
\end{equation}

Für $\kappa = Z \cdot L$ und $c^* =\frac{n}{V}$ ergibt sich aus $\Lambda = \frac{\kappa}{c^*}= \frac{Z \cdot L \cdot V}{n}$ folgende Fehlerfortpflanzung:\\

\begin{equation}
\Delta \Lambda = \sqrt{\left(\frac{Z \cdot V}{n}\right)^2 \cdot {\Delta L}^2 + \left(\frac{Z \cdot L}{n}\right)^2 \cdot \Delta V^2}
\end{equation}

Daraus ergeben sich folgende Fehler für $\Delta L$, welche als Fehlerbalken in die Auftragungen eingetragen werden:\\

\begin{table} [h]
\centering 
\begin{tabular}{|p{4cm}||p{4cm}|p{4cm}|p{4cm}|}
\hline
& 0,1 M & 0,01 M & 0,001 M \\
\hline
$\Delta L$ Essigsäure  & $100,6527 \approx 1\cdot 10$ & $362,21 \approx 4 \cdot 10^2 $& $20776,39 \approx 2 \cdot 10^4$ \\
\hline
$\Delta L$ Kaliumchlorid &$31,5 \approx 3 \cdot 10$ & $315,2 \approx 3 \cdot 10^2$ & $4692,4 \approx 5 \cdot 10^3$\\
\hline
\end{tabular}
\end{table}

\subsection{Fehler für $\Lambda^0$ und $K_{\mathrm{S}}$ aus der Auftragung}

Durch die eingezeichneten Grenzgeraden kann aus der maximalen und minimalen Steigung der absolute Fehler für $\Lambda^0$ bestimmt werden:\\


Bestimmung des Fehlers für $K_{\mathrm{S}}$:\\









\newpage

\section{Literaturverzeichnis}
\begin{flushleft}
1 \quad Gerd Wedler: \emph{Lehrbuch der physikalischen Chemie}, 5. Aufl., WILEY-VCH Verlag GmbH Co. KGaA, Weinheim, \textbf{2004}.\\
\vspace{0,5 cm}
2\quad Götz, Eckold: \emph{Sriptum zur Einführung in die physikalische Chemie}, Institut für physikalische Chemie, Uni Göttingen, \textbf{2015}.\\
\vspace{0,5 cm}
3 \quad \emph{Skriptum für das Praktikum zur Einführung in die Physikalische Chemie}, Institut für physikalische Chemie, Uni Göttingen, \textbf{2015}.\\
\end{flushleft}



\end{document}



